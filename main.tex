\documentclass{article}

\usepackage {framed}
\usepackage [top=3cm, left=3cm, bottom=2cm, right=2cm] {geometry}


\title  {Geometria de Tales e de Pitágoras}
\author {Eduardo Gomes \and João Zambaldi}
\date   {31 de novembro de 2022}


\begin{document}

\maketitle

\section {O espírito Grego} 

Os gregos da antiguidade tinham um desejo de conhecer e capacidade
para absorver todo o conhecimento das culturas que entravam em contato,
os absorviam e construíam em cima disso. Sobre isso Platão disse \footnote{Tradução de um pequeno trecho de Epinomius 987e do inglês, talvez a tradução não seja fidedigna, ena citação que eu queria colocar originalmente está na página 9 \cite{Heath} mas parece que a referência não está correta, de qualquer forma a frase no livro é "Let us take it as an axiom that, whatever the Greeks take from the barbarian, they bring it to fuller perfection" e no livro Heath cita Epinomis 988D como fonte, o que não se verifica com a versão de Epinomis que eu encontrei}.

\begin{framed}
	Sempre que gregos aprendem algo dos bárbaros, isso é 
	transformado em algo mais nobre; [...]
\end{framed}

Falaremos de dois gregos, \emph{Pitágoras de Samos} e \emph{Tales de Mileto},
que viajaram pelo Egito e pela babilônia, aprenderam geometria, aritmética
astronomia e fizeram enormes contribuições, sendo provavelmente os matemáticos
mais conhecidos até os dias de hoje. Sendo assim, começaremos fando sobre o primeiro
matemático da história.


\section {Tales de Mileto}

A tradição têm que Tales de Mileto (624, 548 antes de Cristo) foi o primeiro
matemático e primeiro filosofo, foi Tales a primeira pessoa a provar um teorema,
de fato, a ele são atribuídos em particular os seguintes resultados\footnote{
Página 130 de \cite{Heath}, uma lista similar, diferindo apenas no quinto resultado
é encontrada no capítulo 4 de \cite{Boyer}}
\begin{enumerate}
	\item um círculo é dividido pelo seu diâmetro
	\item os ângulos da base de um triângulo equilátero são congruentes
	\item ângulos opostos pelo vértice são congruentes
	\item se dois triângulos são tais que dois ângulos e uma lado de um deles são
		congruentes respectivamente a dois ângulos e um lado de outro, então
		estes triângulos são congruentes
	\item um triângulo inscrito em uma circunferência, em que um dos lados
		coincide com o diâmetro do círculo é um triângulo retângulo
\end{enumerate}

Existem algumas histórias notáveis em que Tales faz as uso da geometria
das formas mais elegantes possíveis 
\begin{enumerate}
	\item mediu a altura de uma pirâmide usando sua sombra - novamente 
		citando \cite{Heath}, existe várias versões desta história
		a mais simples delas é que Tales percebeu que em um momento
		do dia, a sombra de uma pessoa era igual sua altura, e inferiu
		que a sombra da pirâmide teria também o mesmo comprimento que
		a altura da pirâmide
	\item mediu a distância de uma embarcação eté a praia usando a semelhança
		entre dois triângulos retângulos
\end{enumerate}


\section{Pitágoras de Samos}

O aluno mais notável de Tales foi Pitágoras, como conta Heath na página 4 de \cite{Heath}, a tradição tem que Tales admirado com as habilidades de Pitágoras lhe contou tudo
que sabia e como na época já tinha idade avançada o aconselhou que viajasse ao
Egito para aprender com os sacerdotes. Essa história é bastante controversa,
no já citado capítulo 4 de \cite{Boyer}, Boyer diz que isso é improvável. Mas 
de fato faz parte da tradição.

Pitágoras foi o mestre da \emph{Escola Pitagórica}. Era comum na antiguidade
dar todo o mérito de uma escola ao mestre, e isso é o que aconteceu. Para os
\emph{Pitagóricos},

\begin{center}
\begin{framed}
	tudo é número
\end{framed}
\end{center}

Quando ouvimos isso, não é muito claro o que isso significa, alguns exemplos
fazem esse conceito mais fácil, por exemplo, eles consideravam que os números 
tinham gêneros: os números pares eram femininos e os ímpares eram femininos
de modo que o número $1$ não tinha gênero, o número $5$ era o número do casamento
pois $2 + 3 = 5$, simbolizando a união do primeiro número feminino com o 
primeiro número masculino. O número dez era o número do universo, pois 
$1$ representa o ponto, $2$ pontos distintos determinam uma reta, $3$ pontos 
distintos e não colineares determinam uma reta e $4$ pontos não coplanares 
determinam um tetraedro e $1 + 2 + 3 + 4 = 10$.

A teoria dos números foi originada na escola pitagórica, a definição de paridade
dos pitagóricos também era um pouco diferente, os números pares eram classificados
como ímpares 

\bibliography{refs}
\bibliographystyle{plain}

\end{document}

