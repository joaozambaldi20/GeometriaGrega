\documentclass{article}

\usepackage {framed}
\usepackage [top=3cm, left=3cm, bottom=2cm, right=2cm] {geometry}
\usepackage {tikz}


\title  {Geometria de Tales e de Pitágoras}
\author {Eduardo Gomes \and João Zambaldi}
\date   {Novembro de 2022}


\begin{document}

\maketitle

\section {O espírito Grego} 

Os gregos da antiguidade tinham desejo de conhecer e capacidade
para absorver todo o conhecimento das culturas que entravam em contato.
Uma vez que dominavam a área, logo construíam em cima disso. 
Sobre isso Platão disse \footnote{Tradução de um pequeno trecho de 
Epinomius 987e do inglês, talvez a tradução não seja fidedigna, a 
citação que eu queria colocar originalmente está na página 9 de \cite{Heath} mas 
parece que a referência não está correta, de qualquer forma a frase no
livro é "Let us take it as an axiom that, whatever the Greeks take 
from the barbarian, they bring it to fuller perfection" e no livro 
Heath cita Epinomis 988D como fonte, o que não se verifica com a 
versão de Epinomis que eu encontrei}.

\begin{framed}
	Sempre que gregos aprendem algo dos bárbaros, isso é 
	transformado em algo mais nobre; [...]
\end{framed}

Falaremos de dois gregos, \emph{Pitágoras de Samos} e \emph{Tales de Mileto},
que viajaram pelo Egito e pela babilônia, aprenderam geometria, aritmética
astronomia e fizeram enormes contribuições, sendo provavelmente os matemáticos
mais conhecidos até os dias de hoje. Sendo assim, começaremos fando sobre o primeiro
matemático da história.


\section {Tales de Mileto}

A tradição têm que Tales de Mileto (624, 548 antes de Cristo) foi o primeiro
matemático e primeiro filosofo. Foi Tales a primeira pessoa a provar um teorema.
De fato, a ele são atribuídos em particular os seguintes resultados\footnote{
Página 130 de \cite{Heath}, uma lista similar, diferindo apenas no quinto resultado
é encontrada no capítulo 4 de \cite{Boyer}}
\begin{enumerate}
	\item um círculo é dividido pelo seu diâmetro
	\item os ângulos da base de um triângulo isósceles são congruentes
	\item ângulos opostos pelo vértice são congruentes
	\item se dois triângulos são tais que dois ângulos e uma lado de um deles são
		congruentes respectivamente a dois ângulos e um lado de outro, então
		estes triângulos são congruentes
	\item um triângulo inscrito em uma circunferência, em que um dos lados
		coincide com o diâmetro do círculo é um triângulo retângulo
\end{enumerate}

Existem algumas histórias notáveis em que Tales faz as uso da geometria
das formas mais elegantes possíveis. Duas delas são:
\begin{enumerate}
	\item mediu a altura de uma pirâmide usando sua sombra - novamente 
		citando \cite{Heath}, existem várias versões desta história
		a mais simples delas é que Tales percebeu que em um momento
		do dia, a sombra de uma pessoa era igual sua altura, e inferiu
		que a sombra da pirâmide teria também o mesmo comprimento que
		a altura da pirâmide
	\item mediu a distância de uma embarcação até a praia usando semelhança
		entre dois triângulos retângulos
\end{enumerate}


\section{Pitágoras de Samos}

O aluno mais notável de Tales foi Pitágoras, como conta Heath na página 4 de \cite{Heath}, a tradição tem que Tales admirado com as habilidades de Pitágoras lhe contou tudo
que sabia e como na época já tinha idade avançada o aconselhou que viajasse ao
Egito para aprender com os sacerdotes. Essa história é bastante controversa, e
Boyer diz que isso é improvável\footnote{No capítulo 4 de \cite{Boyer}.}. Mas 
é parte importante da tradição.
Pitágoras foi o mestre da \emph{Escola Pitagórica}. Era comum na antiguidade
dar todo o mérito de uma escola ao mestre, e isso é o que aconteceu. Para os
\emph{Pitagóricos},

\begin{center}
\begin{framed}
	tudo é número
\end{framed}
\end{center}

Quando ouvimos isso, não é muito claro o que significa, alguns exemplos
fazem esse conceito, com alguns exemplos fica mais claro. Eles consideravam que os números 
tinham gêneros: os números pares eram femininos e os ímpares, excetuando o número
$1$, eram masculinos, 
o número $5$ era o número do casamento pois $2 + 3 = 5$, 
simbolizando a união do primeiro número feminino $2$ com o 
primeiro número masculino $3$. 
O número dez era o número do universo, pois 
$1$ representa a dimensão do ponto, 
$2$ representa a dimensão da reta, pois dois pontos distintos determinam uma reta, 
$3$ representa a dimensão do plano, pois três pontos distintos e não colineares determinam um plano e 
$4$ a dimensão do espaço, pois quatro pontos não coplanares determinam um tetraedro.
Assim, 10 que inclui todas as dimensões geométricas, $1 + 2 + 3 + 4 = 10$, 
é o número do universo.

A teoria dos números foi originada na escola pitagórica, eles conheciam
os números triangulares, retangulares, números perfeitos e números amigáveis.

Na geometria os seguintes resultados são atribuidos a escola Pitagórica
\begin{enumerate}
	\item a soma dos ângulos de um triângulo é de dois ângulos retos
	\item o teorema de Pitágoras
	\item os incomenssuráveis
	\item os cinco sólidos regulares
\end{enumerate}

Acreditas-se que a prova de que a soma dos ângulos internos de um triângulo
é de 180 graus se deu por traçar uma reta paralela a um lado de um triângulo
passando pelo vértice que não está nesse lado,

\begin{center}
\begin{tikzpicture}
	\draw 
	(0, 0) node[anchor=north]{B} -- 
	(5, 0) node[anchor=north]{C} -- 
	(2.5, 4.33) node[anchor=south]{A} -- 
	cycle;
	\draw (0, 4.33)  node[anchor=south]{D} -- (5, 4.33)  node[anchor=south]{E};
\end{tikzpicture}
\end{center}
E usando que $DE$ é paralela a $AB$, temos que os ângulos $D \widehat A B$ 
e $A \widehat BC$ são alternos internos. Da mesma forma $E \widehat A C$ e 
$A \widehat C B$ são congruentes. E a soma $D \widehat AB$ $B \widehat AC$ e
$E \widehat AC$ é igual a soma de $B \widehat A C$, $A \widehat BC$ e $B \widehat C A$.


Sobre o \emph{Teorema de Pitágoras} algumas fontes dizem que Pitágoras 
teria sacrificado um gado (em outros relatos 100 gados) em gratidão aos
deuses pela descoberta. As regras da escolas Pitagórica não permitiam
sacrifícios dessa forma e isto é geralmente tido em descredito. Algumas 
fontes citam que o suposto sacrifício teria sido em virtude da descoberta
do triângulo de lados $3, 4, 5$ mas esse triângulo já era conhecido
no Egito e é mais provável que tenha vindo de lá para a Grécia. Ainda 
existem linhas que dizem que o teorema já tenha sido enunciado antes de 
Pitágoras mas foi ele que demonstrou.

Existem algumas linhas de qual demonstração Pitágoras apresentou para o Teorema, 
a primeira é de que a prova foi por decompor os quadrados em partes retangulares e
estas partes retangulares em triângulos como foi feito no livro de Euclides.


\begin{center}
\begin{tikzpicture}
	\draw (0, 0) -- (5, 0) -- (1.8, 2.4) -- cycle;
	\draw (0, 0) -- (0, -5) -- (5, -5) -- (5, 0);
	\draw (1.8, 2.4) -- (-0.6, 4.2) -- (-2.4, 1.8) -- (0,0);
	\draw (1.8, 2.4) -- (4.2, 5.6) -- (7.4, 3.2) -- (5, 0);
	\draw (1.8, 2.4) -- (1.8, -5);
	\draw (1.8, 2.4) -- (0, -5);
	\draw (1.8, 2.4) -- (5, -5);
	\draw (0, 0) -- (7.4, 3.2);
	\draw (5, 0) -- (-2.4, 1.8);
\end{tikzpicture}
\end{center}

No entanto, a ideia mais aceita é a de que Pitágoras tenha usado raciocínio
como das figuras abaixo.

\begin{center}
\begin{tikzpicture}
	\draw (0, 0) -- (5, 0) -- (5, 5) -- (0, 5) -- cycle;
	\draw (4, 0) -- (4, 5);
	\draw (0, 4) -- (5, 4);
	\draw (9, 0) -- (14, 0) -- (14, 5) -- (9, 5) -- cycle;
	\draw (10, 0) -- (9, 4);
	\draw (9, 4) -- (13, 5);
	\draw (13, 5) -- (14, 1);
	\draw (14, 1) -- (10, 0);
\end{tikzpicture}
\end{center}

O raciocínio algébrico de hoje nos permite facilmente mostrar usando
das fórmulas da área de triângulos, retângulos a partir dessa
figura que o teorema de Pitágoras vale, no entanto não foi assim
que se imagina que a demonstração tenha sido feita.


\section{A relação entre o trabalho de Pitágoras e o trabalho de Tales}

Pitágoras e Tales atuaram nas mesmas áreas, em especial: filosofia, geometria e
astronomia. 

Uma interessante conexão entre o trabalho dos dois matemáticos é a soma de que
os ângulos internos de um triângulo é $180^\circ$. Especula-se que se Tales 
realmente mostrou que o triângulo inscrito em um semicírculo é retângulo ele
já estaria em posição para mostrar que a soma dos ângulos internos de um triângulo
é de $180^\circ$. Usando o fato de que a soma dos ângulos agudos de um 
triângulo retângulo é de $90^\circ$, para decompor um triângulo qualquer em dois
triângulos retângulos e então provar que a soma dos ângulos deste triângulo é
também de $180^\circ$.

E para finalizar, eu quero falar sobre duas passagens do comentário de \emph{Proclo}
sobre os \emph{Elementos de Euclides} que
Heath cita \cite{Heath} que eu acho que esclarecem como as contribuições\footnote{
Infelizmente não consegui uma tradução para o português dessas e
fiz um tradução diretamente do livro de Heath \cite{Heath}, do inglês, a
sobre Tales está na página 128 e a sobre Pitágoras na página 141}
de Tales e Pitágoras formam a base da geometria grega. A primeira é sobre
Tales

\begin{framed}
	[...] primeiro foi ao Egito e depois introduziu seus estudos 
	a Grécia. Ele descobriu numerosas proposições próprias e 
	instruiu seus sucessores nos seus métodos de ataque sendo em
	alguns casos mais geral e em outros mais empírico, no sentido
	de simples inspecção ou observação
\end{framed}

e a segunda é sobre Pitágoras

\begin{framed}
	Depois desses [Tales e Ameristus] Pitágoras transformou o estudo
	da geometria em uma educação liberal examinando a ciência dos 
	princípios e provando os teoremas de forma imaterial e intelectual ;
	foi ele quem descobriu a teoria das proporções e a construções das 
	figuras cósmicas 
\end{framed}

Ou seja, além dos incríveis teoremas que eles provaram, Tales e Pitágoras
transformaram a geometria que nasceu no Egito para resolver problemas cotidianos
baseada muito na intuição e no raciocínio prático em uma ciência de fato,
baseada no raciocínio lógico dedutivo que conhecemos hoje,
esse trajeto já tinha sido iniciado no Egito e na Índia mas eles levaram isso 
muito além do que se tem notícia.

\bibliography{refs}
\bibliographystyle{plain}

\end{document}
